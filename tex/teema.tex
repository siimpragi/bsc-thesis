\documentclass[../main.tex]{subfiles}

\begin{document}


\newpage
\section*{Bakalaureusetöö eesmärk ja teema olulisus}

\TODO{Järgnev on info õppejõule aines Eestikeelne kommunikatsioon arvutiteaduses.}

2019. aasta 5. juulil tegi Eesti Interneti Sihtasutus (EIS) nende hallatava tippdomeeni .ee tsoonifaili avalikkusele kättesaadavaks.
Tsoonifail on tekstifail, mis sisaldab andmeid domeeninimede (nt ut.ee, postimees.ee) ning nendega seotud teenuste kohta.
Selleks, et Internetis oleks võimalik ühe domeeniga seotud teenuseid üles leida ja kasutada, tuleb info nende kohta lisada tsoonifaili.
Eesti maatunnusega tippdomeeni tsoonifail on väärt lisa meie riigi avaandmetele.
Maatunnusega tippdomeeni tsooni avaldamine on küllaltki erandlik.
EISile olid eeskujuks rootslased, kes seda 2016. aastal esmakordselt tegid, avaldades nende hallatavad .se ja .nu~\cite{jurgens2019}.

Bakalaureusetöö eesmärk on luua rakendus, mis viib tsoonifailis sisalduvate andmete toel regulaarselt läbi erinevaid uuringuid ning visualiseerib saadud tulemused.
Regulaarselt läbiviidavad uuringud annavad võimaluse tekitada aegrea, vaadelda trende ning hinnata .ee tsooni tervist.

Töö esimene peatükk annab ülevaate domeeninimede süsteemist.

\end{document}
